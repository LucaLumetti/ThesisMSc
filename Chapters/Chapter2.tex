% Chapter 2

\chapter{Segmentation Neural Networks}

\lhead{Chapter 2. \emph{Segmentation Neural Networks}} % This is for the header on each page - perhaps a shortened title
\label{chp:segmentation}
%----------------------------------------------------------------------------------------

\def\:{\hskip0pt} %Definisce un modo veloce per permettere a latex di sillabare correttamente anche parole come 4-connectivity. Il corretto utilizzo è il seguente: 4\:-\:connectivity.


\section{Supervised Learning}
In the medical image segmentation tasks, supervised learning is the most popular
method, where the latest improvements mainly include network backbones, network
blocks, and the design of novel loss functions.

\subsection{Backbone networks}
The encoder-decoder structure is the most popular end-to-end architecture, such
as fully convolutional networks (FNC) like U-Net, Deeplab, and SegNet. The
encoder part aims to extract high-level features from the input image and
project them into a latent space. The decoder part instead restores the
extracted features
from the latent space to the original space size.

\subsubsection{U-Net and V-Net}
In 2015, Ronneberger et al. \cite{ronneberger2015unet} proposed U-Net which has
been widely used for medical image segmentation and many variants have been
proposed also in recent years. The structure of U-Net is symmetrical and the
main scope is to fuse low-level features, which medical images are usually noisy
but show blurred boundaries, with high-level features via skip connections.

U-Net was designed to deal with 2D images but when dealing with medical images,
we usually have to handle 3D images. A straightforward solution was to extract
2D slices from the original image and fed them to the network and stack the
outputs to obtain the final 3D output.
The main drawback of this approach is that we lost the spatial information among
the slices as they are threatened as independent images. Motivated by this idea,
Çiçek et al. \cite{cicek20163dunet} proposed a solution to this problem by using
3D convolutions inside U-Net. This network, named 3D U-Net, includes only three
down-sampling steps because of the high computational cost of 3D Convolutions,
leading to less effective extraction of deep-layer image features.
Milletari et al. \cite{milletari2016vnet} proposed a similar architecture, named
V-Net, which employs more skip connections than U-Net to design a deeper
network. Some Recurrent Neural Network mixed with U-Net has also been proposed
to model the time dependence of image sequences.

\subsubsection{Recurrent Neural Networks}
Gao et al. \cite{gao2018lstm} joined LSTM and CNN to model the temporal
relationships between different brain MRI slices to improve segmentation
accuracy. RNN can capture local and global spatial features of images by
considering the context information relationship. However, in medical image
segmentation, the capture of complete and valid temporal information requires
good medical image quality (e.g. smaller slice thickness and pixel spacing).
Therefore, the design of RNN is uncommon for improving the performance of
medical image segmentation.

\subsubsection{Pseudo-3D Networks}
As stated before, most medical images are 3D images but using 3D convolution
requires a lot of computational resources. Therefore some pseudo-3D segmentation
methods have been proposed. For example, Vu et al. \cite{vu2020pseudo3d} applied
the overlay of adjacent slices as input to the central slice prediction and then
fed the obtained 2D feature map into a standard 2D network.

\subsubsection{Generative Adversarial Networks}
\label{subsubsec:gan}
Another type of networks that has been exploited in medical image segmentation
are Generative Adversarial Networks (GANs) \cite{goodfellow2020gan}, mostly used
in data augmentation by generating new samples. Pollastri et al.
\cite{pollastri2020gan} remodeled two different well-known GANs, Deep
Convolutional GAN, and a Laplacian GAN, to generate both skin lesion images and
their segmentation masks, making the augmentation process extremely
straightforward.
In addition, the incorporation of prior knowledge about
organ shape and position may be crucial for improving the medical
image segmentation effect, where images are corrupted and
thus contain artifacts due to limitations of imaging techniques.
However, there are few works about how to incorporate prior
knowledge into CNN models. As one of the earliest studies in this field, Oktay et al. proposed a novel and general
method to combine a priori knowledge of shape and label
structure into the anatomically constrained neural networks
(ACNN) for medical image analysis tasks. In this way, the neural network training process can be constrained and guided to
make more anatomical and meaningful predictions, especially
in cases where input image data is not sufficiently informative or consistent enough (e.g., missing object boundaries).\\

After proposing U-Net, the encoder-decoder structure
became the most popular structure in medical image segmentation. The design of the network backbone focuses on more
efficient feature extraction in the encoder and feature recovery
and fusion in the decoder to improve segmentation accuracy.

\subsection{Network Function Block}
\subsubsection{Dense connection}
Dense connection is the most popular network block in medical image
segmentation, used to construct a kind of special convolution neural network. The
input of each layer comes from the output of all previous layers in the process of forwarding transmission. Inspired by this design, Guan et al. proposed an
improved U-Net by replacing each sub-block of the network with a form of dense
connection. Although the dense connection helps obtain richer image
features, it often reduces the robustness of feature representation to a certain
extent and increases the number of parameters.

\subsubsection{Inception block}
For CNNs, deep networks often give better performances than shallow ones, but they encounter some new problems such as vanishing gradient, high memory usage, and slow convergence. The inception structure used in GoogleNet overcomes these problems, and for this reason, it has been also used over medical images. Gu et al. proposed CE-Net by introducing the inception structure and atrous convolution to each parallel structure to extract features on a wide reception field. Such complex structure however leads to a difficult model modification.

\subsubsection{Depth separability}
To reduce the computational cost of 3D convolutions and their memory usage,
Howard et al. proposed MobileNet to decompose vanilla convolutions into
depthwise separable convolution and pointwise convolution.
% TODO: aggiungere calcolo del numero di calcoli fatti tra le 3D e le depth
% separable

\subsubsection{Attention}
\par
Attention block can selectively change input or assigns different weights to
input variables according to different importance.\\
\emph{Spatial Attention} block aims
to calculate the feature importance of each pixel in the space domain and extract
the key information of an image. Oktay et al. proposed attention U-Net, where
attention blocks were used to change the output of the encoder before fusing
features from the encoder and the corresponding decoder. The attention block
outputs a gating signal to control the feature importance of pixels at different
spatial positions.\\
Another type of attention block is the \emph{Channel attention},
which utilizes learned global information to emphasize selectively useful
features and suppress useless features. Hu et al. proposed SE-Net
introducing the channel attention to the field of image analysis and winning the
ImageNet Challenge in 2017.\\
Spatial and channel attention mechanisms are the two most popular strategies for
improving feature representation. However, spatial attention ignores the
difference between different channel information and treats each channel equally. On
the contrary, channel attention pools global information directly while
ignoring local information in each channel, which is a relatively rough
operation. Therefore, combining the advantages of two attention mechanisms,
researchers have designed many models based on a \emph{mixed domain attention block}.

\subsection{Loss functions}
In addition to improved segmentation speed and accuracy
by designing the network backbone and the function block, designing new loss
functions also resulted in improvements in subsequent inference-time
segmentation accuracy. Therefore,
a great deal of work has been reported on the design of
suitable loss functions for medical image segmentation tasks.

\subsubsection{Cross Entropy}
The cross-entropy loss has been the most popular loss function. It compares
pixel-wisely the predicted category vector with the real segmentation result
vector and is defined as:
$$
L_{CE} = -\sum_{i=1}^{N} \sum_{j=1}^{C} y_{ij} \log \hat{y}_{ij}
$$
where $y_{ij}$ is the real segmentation result vector, $\hat{y}_{ij}$ is the
predicted segmentation result vector, $n$ is the number of pixels in the image,
and $C$ is the number of categories. The cross-entropy loss is easy to
implement and has been widely used in medical image segmentation tasks. However,
it is not suitable for segmentation tasks with imbalanced data, because it does
not consider the class imbalance problem. Therefore, a \emph{weighted cross
entropy loss} and \emph{balanced cross entropy} have been proposed to solve this
problem, where a $\beta$ hyperparameters are added to the loss function to adjust
the weight of each class.

\subsubsection{Dice Loss}
The Dice coefficient is a popular metric for the evaluation of medical image
segmentation tasks. This metric is the measure of overlap between a segmentation
result and its corresponding ground truth:
$$
DSC = \frac{2 \times |A \cap B|}{|A| + |B|}
$$
where $A$ and $B$ are the segmentation result and the ground truth,
respectively. The \emph{Dice Loss} is then formulated as follows:
$$
L_{DSC} = 1 - \frac{2 \times y \times \hat{y} + 1}{y + \hat{y} + 1}
$$
Here 1 is added to the denominator to avoid the division by zero in the edge
case when both $y$ and $\hat{y}$ are zero. The Dice loss is a good choice even
for uneven samples, however, it easily influences the backpropagation and leads
  to a training difficulty.

\subsubsection{Generalized Dice Loss}
Although Dice Loss can solve the problem of class imbalance to a certain extent,
it does not work for serious class imbalance. To solve this problem, researchers
have proposed a \emph{Generalized Dice Loss} that can be used for both binary
and multi-class segmentation tasks. The generalized Dice loss is defined as:
$$
L_{GDSC} = 1 - \frac{1}{m}\frac{2\sum_{j=1}^{m} \omega_j
\sum_{i=1}^{n}y_{ij}\hat{y}_{ij}}{\sum_{j=1}^{m}\omega_j\sum_{i=1}^{n}(y_{ij} +
\hat{y}_{ij})}
$$
where the weight $\omega_j$ is used to adjust the weight of each class, and $\omega_j = 1/(\sum_{i=1}^{n}p_{ij})^2$.

\subsubsection{Boundary Loss}
Another approach that solves the problem of class imbalance has been proposed by
Kervadec et al. for the task of brain lesion segmentation. They proposed a Boundary Loss which aims to minimize the distance between segmented boundaries
and labeled boundaries. Results show that the combination of the Dice loss and
the boundary loss is superior to the single ones. The composite loss is defined
as:
$$
L = \alpha L_{DSC} + (1 - \alpha) L_{B}
$$
where the Boundary Loss $L_{B}$ for a binary segmentation is defined as:
$$
L_{B} = \sum \phi(y) \times \hat{y}
$$
where $\phi(y)$ is the signed distance function applied to the real segmentation $y$.

\par
For medical image segmentation, the improvement of loss mainly focuses on the
problem of segmentation of small objects in a large background (the problem of
class imbalance). Chen et al. proposed a new loss function by applying
traditional active contour energy minimization to convolutional neural networks,
Li et al. proposed a new regularization term to improve the cross-entropy loss
function, and Karimi et al. proposed a loss function based on Hausdorff
distance (HD). Besides, there is still a lot of work trying to deal with this
problem by adding penalties to lose functions or changing the optimization
strategy according to specific tasks.\\
In many medical image segmentation tasks, there are often only one or two
targets in an image, and the pixel ratio of targets is sometimes small, which
makes network training difficult. Therefore, to improve network training and
segmentation accuracy, it is easier to focus on smaller targets by changing loss
functions than to change the network structure. However, the design of loss
functions is highly task-specific, so we need to analyze carefully task
requirements, and then design reasonable and available loss functions.

\section{Weakly Supervised and Unsupervised Segmentation}
Although convolutional neural networks show goods performances for medical
image segmentation, results seriously depend on high-quality
labels. It is rare to build many datasets with many high-quality labels,
especially in the field of medical image analysis, since data acquisition and
labeling often incur high costs. Therefore, a lot of studies on incomplete or
imperfect datasets are reported. Unsupervised learning is a very important approach that improves the performance of medical image segmentation. In this
section, we will introduce weakly supervised learning, a method that makes
use of unsupervised learning for unlabelled data in combination with supervised
learning with labeled data.

\subsection{Data Augmentation}
Performing data augmentation is mandatory in absence of a largely labeled dataset, and is still considered good practice even when we have enough data. However, new data generated with this method produce images that are highly correlated with the original images.

\subsubsection{Traditional methods}
With traditional methods, we refer to all the computer vision techniques such as adding/removing noise, changing brightness, saturation, contrast, colors, and changing the image layout with rotations, distortion, scaling, etc. These techniques are still very used today, usually combining them together with random parameters. Some of these algorithms may require a non-negligible computational cost thus usually performed before the training procedure.

\subsubsection{Conditional Generative Adversarial Networks}
As already described in Section \ref{subsubsec:gan}, GAN and its variants have been widely for data augmentation. In particular, cGANs are often used in combination with standard GANs to generate labels relative to a given synthetic image.

\subsection{Transfer Learning}
Pre-trained model parameters are often used to initialize a new model,
transfer learning can achieve fast training for data with limited labels. The
most popular approach is to use a model pretrained on ImageNet before performing
the training on the medical data. Experiments demonstrated that this approach is
useful as it improves the accuracy of segmentations. However, domain
adaptation may be a problem when applying models trained over natural images to
medical image analysis tasks. Moreover, these methods are hardly applicable to
3D medical image analysis because such pre-trained models rely on 2D datasets.\\
Hatamizadeh et al. recently proposed an unsupervised approach to pre-train a
given model by relying only on unannotated medical images of the same domain of
the main tasks. In practice, the network is trained to perform a variety of tasks
such as image reconstruction, classification of the rotation applied to the
original image, etc. which can be performed in an unsupervised manner. Later,
these tasks heads of the network are detached and the training on the main task
is performed. Such pretraining aims to train the network to learn how to extract
high-level features from a specific type of medical images, such as CT or MRI.\\
Also, Cipriano et al. recently proposed a pre-training approach based on sparse labels.
This type of labels are way easier to obtain but is not as accurate as the real dense labels. They fed these labels to the network together with the input
image for a given number of steps, then used the learned parameters to train the
network to produce the segmentation by relying only on the input, without the
sparse labels.


\section{Current direction of research}
Until now we described the most popular network structures and loss functions
for medical image segmentation tasks that were proposed and used up to date.
Since the rise in the popularity of vision transformers and graph neural networks,
some novelty architectures are being proposed in medical imaging. Results
obtained are still not as good as those obtained with traditional architectures,
aside from some really specific tasks or datasets, but they are still interesting and
promising.\\

\subsection{Network Architecture Search}
The design process of network architecture is a very time-consuming task, and
it is often difficult to find the best architecture for a given task. Therefore,
many researchers have proposed methods to automate the design of network
architectures. Such methods, named NAS (Network Architecture Search), focus on
the \emph{search space}, \emph{search strategy}, and \emph{performance
estimation}.
The search space is a candidate collection of network structures to be searched.
The search space is divided into a global search space that represents the
search for the entire network structure, and a cell-based search space that
searches only a few small structures that are assembled into a complete large network by the ways of stacking and stitching. The search strategy aims to find
the optimal network structure as fast as possible in search spaces. Popular
search strategies are often grouped into three categories, reinforcement-based
learning, evolutionary algorithms, and gradients. The performance estimation
strategy is the process of assessing how well the network structure performs on
target datasets. For NAS techniques, researchers pay more attention to the
improvement of search strategies since search space and performance estimation
methods are usually rarely changed.\\
Isensee et al. argued that too much manual adjustment on network structure could
lead to over-fitting for a given dataset, and therefore proposed a medical image
segmentation framework no-newUNet (nnU-Net) that adapts itself to any new
dataset. The nnUNet automatically adjusts all hyperparameters according to the
properties of the given dataset without manual intervention. Therefore, the
nnU-Net only relies on vanilla 2D UNet, 3D UNet, UNet cascade, and a robust
training scheme. It focuses on the stage of pre-processing (resampling and
normalization), training (loss, optimizer settings, data augmentation),
inference (patch-based strategies, test-time-augmentations integration, model
integration, etc.), and post-processing (e.g., enhanced single pass domain). In
practical applications, the improvements in network structure design usually
depend on experiences without adequate interpretability theory support,
Moreover, more complex network models indicate a higher risk of over-fitting.


\subsection{Graph Convolutional Neural Network}
Graph Convolutional Neural Network (GCN) is a type of neural network that
utilizes graph structure to process data. In practice, the Euclidean space of the
image can be converted into graphs that can be modeled using GCN.

Gao et al. designed a new graph pooling (gPool) and graph unpooling (gUnpool)
operation based on GCN and proposed an encoder-decoder model namely graph U-Net.
The graph U-Net achieves better performance than popular UNets by adding a small
number of parameters. In contrast to traditional convolutional neural networks
where deeper is better, the performance of the graph U-Net cannot be improved by
increasing the depth of networks when the value of depth exceeds 4. However, the
graph U-Net shows a stronger capability of feature encoding than popular U-Nets
when the value of depth is smaller or equivalent to 4.

Yang et al. proposed the end-to-end conditional partial residual plot
convolutional network CPR-GCN for automatic anatomical marking of coronary
arteries. Authors showed that the GCN-based approach provided better performance
and stronger robustness than traditional and recent depth learning-based
approaches. Results from these GCNs in medical image segmentations are
promising, as the graph structure has high data representation efficiency and
a strong capability of feature encoding.

\subsection{Interpretable Shape Attentive Neural Network}
Currently, many deep learning algorithms tend to make judgments by using
"memorized`` models that approximately fit input data. As a result, these
algorithms cannot be explained sufficiently and give convincing evidence for
each specific prediction. Therefore, the study of the interpretability of deep
neural networks is a hot topic at present. Sun et al. proposed the SAU-Net which
focuses on the interpretability and robustness of models. The proposed
architecture attempts to address the problem of poor edge segmentation accuracy
in medical images by using a secondary shape stream. Especially, the shape stream
and the regular texture stream can capture rich shape-dependent information in
parallel. Furthermore, both spatial and channel attention mechanisms are used for
the decoder to explain the learning capability of models at each resolution of
U-Net. Finally, by extracting the learned shape and spatial attention maps, we
can interpret the highly activated regions of each decoder block. The learned
shape maps can be used to infer the correct shapes of interesting categories learned
by the model. The SAU-Net can learn robust shape features of objects via
the gated shape stream and is also more interpretable than previous works via
built-in saliency maps using attention.

\subsection{Vision Transformer}
Recently, transformer-based architectures have become very popular and replaced the convolutional operator and use self-attention modules to compose entire
encoderdecoder structures that can encode long-range dependencies.
It has been a great success in the field of natural language processing.
Dosovitskiy et al. proposed Vision Transformer (ViT) that can classify
images directly using the Transformer.
Recently, a large number of researchers have applied the transformer to medical
image segmentation. CNNs have a comparative advantage in extracting the
underlying features. These low-level features form the key points, lines, and
some basic image structures at the patch level. However, when we detect these
basic visual elements, the higher-level visual semantic information is often
more concerned with how these elements relate to each other to form an object,
and how the spatial location of objects relates to each other to form the scene.
At present, the transformer is more natural and effective in dealing with the
relationships between these elements. However, if all the convolutional
operators in CV tasks are replaced by Transformer, it may suffer from many
problems, such as high computational cost and memory usage. From existing research, the combination of Transformer and CNNs may lead to better results.
Recently, Chen et al. proposed a U-Net shaped network, where the encoder was
made of ViT only while the decoder was fully convolutional.


\chapter{Conclusions and future work}

\lhead{\textit{Conclusions and Future Work}} % This is for the header on each page - perhaps a shortened title

%----------------------------------------------------------------------------------------

\section{Conclusion}
\label{sec:conclusion}
The work described in this thesis has been concerned with the analysis of
extremely novel techniques for the segmentation of 3D medical images, applied to
our specific problem of segmenting the Inferior Alveolar Canal (IAC) in
cone-beam CT (CBCT) images.
After presenting the state-of-the-art in the field of image segmentation, the
work focused on improving the already existing codebase of PosPadUNet3D by
making the code more maintainable, optimizing some process of the pipeline and
by adding new features.
Finally, novel architecture proposed were implemented using PyTorch and TorchIO,
even from scratch when no code were provided in the relative paper, tested and
compared with the work carried out by Cipriano \etal.

Even if no substantial improvement regarding the metrics has been made, this
work resulted in a improvement of the code for future work and in a contribution
to the TorchIO library.

\section{Recommendation for Future Work}
All the segmentation outputs that we've got with the tested networks had in
common the lack of being a single tube for each canal that we wanted to segment.
This problem can be overcome by using shape model and asking to the network to
output the parameters for such model instead of a $H \times W \times D$
segmentation map.
Some novel loss that care about the topology of the output have been proposed,
but has not been yet widely used by the research community and hence it has not
been a priority to try during this work.
Lastly, transformers are always being more and more popular in the field of
computer vision hence the results we've got with our tests can still have some
room for improvements by getting inspired by novel approach used also in other
field such as NLP.
